{\bfseries \-Función del programa}\-: 

\-Lee un grafo que representa un mapa en el plano y lo colorea de manera que no haya dos nodos conectados que tengan el mismo color utilizando para ello el menor número de colores posible y como máximo un número dado de colores

{\bfseries \-Datos de entrada}\-: 

\-El fichero de datos de entrada consta de una linea que indica el número máximo de colores que se pueden usar, una segunda linea con el número de nodos del grafo y a continuación una matriz de adyacencia que describe el grafo problema

{\bfseries \-Datos de salida}\-: 

\-El programa imprime una lista de nodos con el color que se ha usado para cada uno de ellos

{\bfseries \-Uso}\-: 

\$java coloreado \mbox{[}-\/t\mbox{]}\mbox{[}-\/h\mbox{]} \mbox{[}fichero\-\_\-entrada\mbox{]} \mbox{[}fichero\-\_\-salida\mbox{]} 

\begin{DoxyAuthor}{\-Author}
\-Leonardo \-Garma 
\end{DoxyAuthor}
\begin{DoxyVersion}{\-Version}
0.\-2.\-0 09/01/2015 
\end{DoxyVersion}
